\section{Conclusions}

We provided the definitions of several types of machine intelligence, distinguishing between general and strong AI and showing that each one does not imply the other. We then described Searle's Chinese room argument against strong AI and provided counterarguments to it based on the works of Harnad in \cite{harnad1989minds} and Rey in \cite{rey1986s}. Our belief is that the program provided in the room is not the same as the one necessary for understanding, and even if it were, the person inside the room would still not be able to understand, as that program would not run directly on their brain. Furthermore, we argue that Searle fails to account for the possibly multi-program nature of the brain. We addressed the internalized version of the Chinese room, resolving an apparent contradiction in our claims. Finally, we proved that the Chinese room does not question in the slightest the validity of strong AI as a functionalist theory of the mind. There is still no evidence that modern computers cannot attain consciousness.
