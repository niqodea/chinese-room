\section{On Simulations}
\label{sec:simulations}

Harnad presents multiple arguments to challenge the Chinese room. One of them is the ``Simulation versus Implementation'' argument, where he argues that ``Searle fails to distinguish between the simulation of a mechanism, which is only the formal testing of a theory, and the implementation of a mechanism, which does duplicate causal powers''~\cite[p.~12]{harnad1989minds}.

Searle seems to acknowledge to an extent the importance of programs and simulations in cognitive science, stating that he has ``no objection to the claims of [the] weak AI [community]'', as in ``the principal value of the computer in the study of the mind [being] that it gives us a very powerful tool''~\cite[p.~417]{searle1980minds}. However, unlike Searle, Harnad provides a detailed analysis of the term ``simulation''.

Harnad contraposes the concept of simulation to that of implementation. He imagines a computer simulation of flight, taking into consideration all known aerodynamic factors, where models of planes can be tested. He then supposes that, after finding a plane that can fly in the simulation, the same plane is reproduced in the real world and is shown to also be able to fly. From this example, we can infer that simulation is abstract, theoretical, and formal, while implementation is concrete, practical, and physical. Nonetheless, ``both contain the relevant theoretical information, the relevant causal principles''~\cite[p.~2]{harnad1989minds}.

The idea of simulation is then applied to analyze Searle's Chinese room. Inside the room, a person simulates \textit{something} and produces behavior that is the same as that of a person understanding Chinese. That \textit{something} the person is simulating is, according to Harnad, not understanding Chinese, but rather \textit{the simulation of understanding Chinese}. In summary, ``Searle's "simulation" only simulates simulation rather than implementation''~\cite[p.~12]{harnad1989minds}. Finally, because ``[t]he simulation of understanding Chinese does not understand Chinese any more than the simulation of flying flies''~\cite[p.~4]{harnad1989minds}, it is only natural for the person inside the room to not understand a word of Chinese. In conclusion, Harnad's argument recognizes Searle's Chinese room as a simulation and proves the inadequacy of the thought experiment.

While we believe in the validity of Harnad's conclusions, we believe the argument to present many issues. First, we argue that Harnad confuses the simulation with the simulation of the simulation when explaining why the person does not understand Chinese. In theory, one could only claim that \textit{the simulation of the simulation of understanding Chinese does not simulate understanding Chinese}. Still, we argue that there is no reason to believe that such ``nested simulation'' would result in understanding Chinese either.

Furthermore, we do not believe that what the person is simulating is a simulation of understanding Chinese. A simulation would need to be informative about the implementation; however, Searle does not provide any detail on the type of program the person is simulating in the room\footnote{We refer to the original Chinese room formulation} other than that the obtained behavior is the same as the behavior of a Chinese speaker; the use of a mere (astronomically large) look-up table could replicate such behavior without providing any insight on how the Chinese speaker understands.

Finally, even if the program inside the room were to be the program of understanding Chinese\footnote{We define the program of $X$ as the program that, when run by a machine, causes $X$.} (supposing such program exists), we argue that the person inside the room would still not understand Chinese: they would not \textit{implement} the program, but rather \textit{simulate} the program, and the simulation, as we have seen before, does not understand. We thus consider the Chinese room to be flawed either way. We believe that the problem resides in the person, which is only capable of simulating the program provided in the room.

The three following questions naturally arise from our considerations.
\begin{enumerate}
    \item What is the person inside Searle's Chinese room actually simulating?
    \item What does it mean for a person to \textit{implement} a program?
    \item Why can the person inside the room only \textit{simulate} the program provided in the room?
\end{enumerate}
We address these points by drawing parallels with the work of Georges Rey~\cite{rey1986s} in section \ref{sec:functionalism}.
